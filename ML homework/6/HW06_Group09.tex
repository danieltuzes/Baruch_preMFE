\documentclass{article}
\usepackage{amsmath} % for matrices
\usepackage{amssymb}
\usepackage{enumitem}
\usepackage{booktabs} % For better looking tables
\usepackage{graphicx} % For including images
\usepackage{caption}  % (Optional) For customizing captions
\usepackage{siunitx}
\usepackage{pdfpages}
\usepackage{physics} % For abs
\PassOptionsToPackage{hyphens}{url}\usepackage{hyperref}

\title{Baruch ML HW 6}
\author{Annie Yi, Daniel Tuzes, group 9}

\begin{document}
\maketitle

% insert table of contents here
\tableofcontents

\section{Stock return classification}
In this exercise, we will collect financial data on various companies,
and potentially, further inputs that can affect stock prices to predict
whether a stock will go up, stay the same, or go down in the next period.

We then define a simple classification model,
the details of which we obtain from the Baruch ME pre-MFE course.
We label the data, train the model, and evaluate its performance.

Our goal is to show how a multi-dimenional,
multi-label classification problem
can be defined, trained, and used for prediction. Our hope to predict
market movement with any signicant accuracy is low.
    [But whou wouldn't want to be rich?]

\subsection{Feature selection}
For brainstorming, we considered the features in the appendix.
We concluded in using the following features:

\begin{enumerate}
    \item The daily returns are used with different
          averaging windows. While the the weight within the
          window is constant, some more complicated averaging
          methods would be using decreasing weights as time
          difference increases. We can expact that if we went
          with a single averaged parameter, we would need to use
          a narrower window for daily return prediction,
          and a wider window for the monthly return prediction.

          We used a window size of 5 for daily and weekly returns,
          and window size of 3 for monthly returns. When calculating
          the weekly and monthly returns, we took the average over
          that time period, so the frequency of the data changed
          to weekly and monthly. This also means that the window size
          increased to 5 weeks and 3 months.
    \item Tha volatility is also used, where similarly,
          different prediction horizon required different
          time ranges. We took the sample's standard deviation
          over the same time windows as above.
    \item The trading volume is used, with the same
          considerations as above.
    \item As we were not familiar with technical indicators,
          we omitted them from our list.
    \item Price to earning is a widely used metric,
          and we wanted to incorporate it, however,
          it would require earnings data and total number of shares too,
          and our free source yfinance had no data before 2020 for some
          of the stocks.
    \item we selected stocks which pay little to no dividends,
          so we omitted the dividend yield.
    \item We considered the interest rate, however,
          we selected a time frame where the interest rate
          was constant low, so we omitted it.
          This raised a serious limitation on the training
          data, as the interest was low only between 2009 and 2015.
          To collect enough data, later we extended the time frame
          till 2022, when the interest rate was slightly elevated.
          See fig \ref{fig:interest} for the interest rate
          between 2009 and 2022.

\end{enumerate}

\begin{figure}[h]
    \centering
    \includegraphics[width=0.7\textwidth]{IR.png}
    \caption{The interest rate between 2009 and 2022. Note that
        the interest rate was constant low between 2009 and 2015
        and slightly elevated between 2015 and 2022. Overall,
        the interest rate was low during this period compared
        to the historical average. Grey zones indicate recessions.
        Source: \url{https://www.macrotrends.net/2015/fed-funds-rate-historical-chart}}
    \label{fig:interest}
\end{figure}


\subsection{Data collection}
We restrictred the model to features that are only stock specific and can be calculated from
previous stock prices and volumes. See the collected data in fig \ref{fig:amazon_scaled_data}.
Chartists were people who believed that
the stock price movement can be predicted by previous data patterns.
In this case, we are doing the same, on a small set of features,
but we are using a machine learning model to predict the future price movement.

For code of data collection and preparation, see the attached notebook in the appendix.

\subsubsection{Training data and test data}
We used the data from 2009 to 2020 for training, and the data from 2020 to 2022 for testing.

\begin{figure}[ht]
    \centering
    \includegraphics[width=\linewidth]{amzn_stock_data_trends.pdf}
    \caption{Stock prices for Amazon, their volatility and trading volume.
        Weekly sampling and moving average window of size 5 is used.
        Normalization is applied, so that the standard deviation of the
        different features and the percentage change of the stock price
        is 1.}
    \label{fig:amazon_scaled_data}
\end{figure}

\subsection{Labeling}
We labeled the data based on the stock price movement.
For every time point $t_i$,
we calcualted moving averages and volatility over the past few data points.
We ensure that the time point $i$ is not included in the calculation,
see technical details in the notebook and how we shifted the data by 1 cell.

We defined a symmetric threshold $\tau$, and labeled the data as follows:
\begin{itemize}
    \item pozitive, $y_i = 1$ if $p_{t_i} > \tau$
    \item neutral, $y_i = 0$ if $\abs{p_{t_i}} < \tau$
    \item negative, $y_i = -1$ if $p_{t_i} < - \tau$,
\end{itemize}

where $p_{t_i}$ is the stock price movement (Pct\_Return) at time $t_i$.

We defined the value for the threshold so that
the number of samples in the neutral band is $1/3$ of the total number of samples.
The number of samples in the positive and negative classes are not necessarily equal.
The distribution of the percentage daily return and the result of the labeling
is shown in fig \ref{fig:amazon_labelled_pct_return}.

\begin{figure}[ht]
    \centering
    \includegraphics[width=\linewidth]{amzn_future_pct_return_distribution.pdf}
    \caption{Percentage daily return distribution for Amazon.
        The threshold is set so that the number of samples in the neutral band
        is $1/3$ of the total number of samples. The difference between the
        positive and negative classes is a sign of the asymmetry of the
        stock price movements.}
    \label{fig:amazon_labelled_pct_return}
\end{figure}

\subsection{Model training}
We calculate the center of mass for different classes,
\[{\mu _k} = \frac{1}{{{N_k}}}\sum\limits_{i = 1}^{{N_k}} {{x_i}} \]
where $N_k$ is the number of samples in class $k$,
and $x_i$ is the feature vector of sample $i$.

We calculate the standard deviation as
\[{\sigma ^2} = \frac{1}{{N - K}}\sum\limits_{k = 1}^K {\sum\limits_{i = 1}^N {{{\left( {{x_i} - {\mu _k}} \right)}^2}} } \]
Here we supposed that the features are independent, so the covariance matrix is diagonal,
hence the higher-dimensional Euclidean distance can be calculated as the sum of the
squared differences of the features,
and the standard deviation can be calculated
as the sum of the standard deviations of the features.

The training outcome for Amazon is shown in table \ref{tab:com_std} below.
We can see that the features share a similar center of mass for the different classes,
and the standard deviation is similar for all features. The concern that the features
are not distinguishing the classes well enough is further supported by the
a 3D scatter plot in fig \ref{fig:3d_scatter_plot}.



\begin{table}
    \caption{Center of Mass and Standard Deviation for Each Feature for AMZN. CoM: center of mass}
    \label{tab:com_std}
    \begin{tabular}{lrrrr}
        \toprule
                        & CoM\_-1 & CoM\_0 & CoM\_1 & Std  \\
        Feature         &         &        &        &      \\
        \midrule
        Pct\_Return\_MA & 0.17    & 0.24   & 0.10   & 0.96 \\
        Volume\_MA      & 2.10    & 1.81   & 2.05   & 1.04 \\
        Volatility      & 1.55    & 1.30   & 1.52   & 0.98 \\
        \bottomrule
    \end{tabular}
\end{table}

% include the pdf of 3d_scatter_plot.pdf
\begin{figure}[ht]
    \centering
    \includegraphics[width=1\linewidth]{3d_scatter_plot.pdf}
    \caption{3D scatter plot of the features for Amazon. The features are not
        distinguishing the classes well enough.}
    \label{fig:3d_scatter_plot}
\end{figure}

\subsection{Prediction and evaluation}
To classify a new sample, we calculate the Euclidean distance and the relative frequency,
and use the Bayes formula to calculate the probability of the sample belonging to
a class,
\[P\left( {y = k|x} \right) = \frac{{P\left( {y = k} \right)P\left( {x|y = k} \right)}}{{\sum\limits_{k = 1}^K {P\left( {y = k} \right)P\left( {x|y = k} \right)} }}\]

where $P(y=k)$ is the relative frequency of class $k$,
and $P(x|y=k)$ is the probability of the sample $x$ given that it belongs to class $k$.

We then take the class with the highest probability as the prediction.
Note that denominator is the same for all classes, so we can skip it for the
classification. For the conditional probability, we use the Gaussian distribution,
\[P\left( {x|y = k} \right) \propto \exp \left( { - \frac{{{{\left( {x - {\mu _k}} \right)}^2}}}{{2{\sigma}^2}}} \right)\]
where $\mu_k$ is the center of mass for class $k$,
and $\sigma$ is the standard deviation, $\sigma^2 = \sum\limits_{i = 1}^N {\sigma _i^2}$,
and where the normalization constant is omitted, as it is the same for all classes.

The results are  shown in table \ref{tab:results} below. We can see that the accuracy
is around $1/3$, which is not better than random guessing.

\begin{table}
    \caption{Stock Classification Accuracy Results for Different Frequencies}
    \label{tab:results}
    \begin{tabular}{llll}
        \toprule
              & Accuracy\_D & Accuracy\_W & Accuracy\_M \\
        \midrule
        AMZN  & 0.36        & 0.30        & 0.25        \\
        GOOG  & 0.38        & 0.36        & 0.38        \\
        TSLA  & 0.31        & 0.30        & 0.25        \\
        CRM   & 0.38        & 0.30        & 0.25        \\
        ADBE  & 0.38        & 0.43        & 0.38        \\
        BIIB  & 0.35        & 0.34        & 0.17        \\
        ISRG  & 0.37        & 0.34        & 0.42        \\
        VRTX  & 0.40        & 0.32        & 0.21        \\
        REGN  & 0.39        & 0.37        & 0.33        \\
        ILMN  & 0.36        & 0.34        & 0.21        \\
        EBAY  & 0.37        & 0.26        & 0.38        \\
        BIDU  & 0.30        & 0.30        & 0.33        \\
        NFLX  & 0.41        & 0.30        & 0.29        \\
        BKNG  & 0.36        & 0.39        & 0.33        \\
        LULU  & 0.37        & 0.35        & 0.33        \\
        ADSK  & 0.31        & 0.30        & 0.46        \\
        BRK-B & 0.37        & 0.39        & 0.54        \\
        BMRN  & 0.39        & 0.31        & 0.29        \\
        ALGN  & 0.37        & 0.33        & 0.29        \\
        META  & 0.24        & 0.32        & 0.50        \\
        EA    & 0.38        & 0.37        & 0.25        \\
        \bottomrule
    \end{tabular}
\end{table}

\subsection{Improvement possibilities}

\subsubsection*{Number of features}
Using only 3 features should be enough, if a well-founded model is used,
and if the market didn't eliminate any "alpha" as soon as there is a hint of it.
While in general, using more features should improve the model,
we wouldn't expect a significant improvement in the accuracy
because of the financial benefit for the ones who could first find them and use them.
Using only 3 of them gave the opportunity to keep the model, code and evaluation simple,
and it was even possible to visualize the data in 3D (or at least a 2D projection of it).

By adding more features, we would also need to consider regularization,
as the model could overfit the data. For this, we could look for regularization methods
that tells the relevance of the features.

\subsubsection*{Averaging windows and shift}
The averages we are taking is rolling averages, which means that each of the
points in the average is equally weighted. We could use an exponential average,
where the weights decrease exponentially as we go back in time.

Some Chartists believe that there is a predictable periodicity in the stock prices,
so we could use a shift in the data, so that the model could learn the periodicity.


\subsubsection*{Handling non-independent features}
The assumption that the features are independent is not true in general.
One possible way for improving the model is to use a covariance matrix,
\[{\Sigma} = \frac{1}{{N - K}}\sum\limits_{k = 1}^K {\sum\limits_{i = 1}^N {\left( {{x_i} - {\mu _k}} \right){{\left( {{x_i} - {\mu _k}} \right)}^T}} } \]
And then use the multivariate Gaussian distribution for the conditional probability,
\[P\left( {x|y = k} \right) \propto \exp \left( { - \frac{1}{2}{{\left( {x - {\mu _k}} \right)}^T}{\Sigma ^{ - 1}}\left( {x - {\mu _k}} \right)} \right)\]
where the normalization constant is omitted, as it is the same for all classes.


% appendix

\section{Appendix}
\subsection{Brainstorming on the list of features}
\small

Price-Based Features:

\begin{itemize}
    \item Daily Returns: The percentage change in stock price from one day to the next.
    \item Moving Averages: Simple Moving Average (SMA) and Exponential Moving Average (EMA) over different periods (e.g., 5-day, 10-day, 20-day).
    \item Volatility: Standard deviation of daily returns over a certain period. Relative Strength Index (RSI): Measures the speed and change of price movements.
    \item Bollinger Bands: Uses moving averages and standard deviations to identify overbought or oversold conditions.
\end{itemize}

Volume-Based Features:

\begin{itemize}
    \item Trading Volume: The number of shares traded in a given period.
    \item Volume Moving Averages: Similar to price moving averages but applied to trading volume.
    \item On-Balance Volume (OBV): A cumulative total of volume that adds or subtracts volume based on price movement.
\end{itemize}

Technical Indicators:

\begin{itemize}
    \item MACD (Moving Average Convergence Divergence): Shows the relationship between two moving averages of a stock’s price.
    \item Stochastic Oscillator: Compares a particular closing price of a security to a range of its prices over a certain period.
    \item Average True Range (ATR): Measures market volatility by decomposing the entire range of an asset price for that period.
\end{itemize}

Fundamental Features:

\begin{itemize}
    \item Earnings Reports: Quarterly earnings, earnings per share (EPS), and revenue growth.
    \item Financial Ratios: Price-to-Earnings (P/E) ratio, Price-to-Book (P/B) ratio, and Debt-to-Equity ratio.
    \item Dividend Yield: The dividend income relative to the stock price.
\end{itemize}

Macro-Economic Indicators:

\begin{itemize}
    \item Interest Rates: Changes in interest rates can affect stock prices.
    \item Economic Indicators: GDP growth rate, unemployment rate, and inflation rate.
\end{itemize}

Time-Based Features:

\begin{itemize}
    \item Day of the Week: Some stocks exhibit patterns based on the day of the week.
    \item Seasonality: Monthly or quarterly trends.
\end{itemize}

\subsection{Calculation notebook}
The notebook used for the homework is attached.
The document was prepared with the commands
\begin{verbatim}
jupyter nbconvert --to latex fitting.ipynb
pdflatex fitting.tex
\end{verbatim}

\includepdf[pages={1-}]{fitting.pdf}

\end{document}
