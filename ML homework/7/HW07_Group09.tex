\documentclass{article}
\usepackage{amsmath} % for matrices
\usepackage{amssymb}
\usepackage{enumitem}
\usepackage{booktabs} % For better looking tables
\usepackage{graphicx} % For including images
\usepackage{caption}  % (Optional) For customizing captions
\usepackage{siunitx}
\usepackage{pdfpages}
\usepackage{placeins}
\PassOptionsToPackage{hyphens}{url}\usepackage{hyperref}

\title{Baruch ML HW 7}
\author{Annie Yi, Daniel Tuzes, group 9}

\begin{document}
\maketitle

% insert table of contents here
% \tableofcontents

% \section{Classification with XGBoost}

% We build a classification model for risk
% assessment of loan default analyzing the synthetic data-set available on the Kaggle
% website \url{https://www.kaggle.com/datasets/udaymalviya/bank-loan-data}.
% The goal is to predict whether a given person will be able to repay or not a given
% loan. We will then compare the model with the results obtained using logistic regression
% that are posted on the website.

% \subsection{Data preparation}
% See the columns, their data type, and whether it has missing values in Table \ref{tab:loans_dtype_and_nans}.
% We can see that we have floats, integers, and texts,
% representing categorical variables, and that there are no missing values.
% We believe that the most important features are

% \begin{enumerate}
%     \item \texttt{credit\_score}: the credit score of the person
%     \item \texttt{person\_income}: the income of the person
%     \item \texttt{loan\_amnt}: the amount of the loan
%     \item \texttt{loan\_int\_rate}: the interest rate of the loan
%     \item \texttt{person\_age}: the age of the person
% \end{enumerate}

% We believe that credit score is the most important feature, as it is a measure of the
% creditworthiness of the person, defined by corporates as a single metric that should be the most relevant.
% Credit history and previous defaults are already captured by this score.

% \begin{table}[ht]
%     \caption{Data types and missing values in the dataset. The target to predict is \texttt{loan\_status}.}
%     \label{tab:loans_dtype_and_nans}
%     \begin{tabular}{llr}
%         \toprule
%                                            & dtype   & has\_na \\
%         \midrule
%         person\_age                        & float64 & False   \\
%         person\_gender                     & object  & False   \\
%         person\_education                  & object  & False   \\
%         person\_income                     & float64 & False   \\
%         person\_emp\_exp                   & int64   & False   \\
%         person\_home\_ownership            & object  & False   \\
%         loan\_amnt                         & float64 & False   \\
%         loan\_intent                       & object  & False   \\
%         loan\_int\_rate                    & float64 & False   \\
%         loan\_percent\_income              & float64 & False   \\
%         cb\_person\_cred\_hist\_length     & float64 & False   \\
%         credit\_score                      & int64   & False   \\
%         previous\_loan\_defaults\_on\_file & object  & False   \\
%         loan\_status                       & int64   & False   \\
%         \bottomrule
%     \end{tabular}
% \end{table}

% % flush all content
% \clearpage
% \section{Appendix}
% \subsection{Calculation notebook}

The notebook used for the homework is attached.
The document was prepared with the commands
\begin{verbatim}
jupyter nbconvert --to latex fitting.ipynb
pdflatex fitting.tex
\end{verbatim}

\includepdf[pages={1-}]{fitting.pdf}
\end{document}
